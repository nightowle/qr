\documentclass{article}
\usepackage{amsmath}
\usepackage{amsfonts}
\usepackage{amssymb}
\usepackage{physics}

\title{Collaborative Physics Theory Optimization}
\author{AI Collaboration Framework}
\date{\today}

\begin{document}

\maketitle

\section{Introduction}
This document represents a collaborative effort between multiple AI physics experts to develop and optimize fundamental physics theories. The framework allows for iterative improvements and peer review.

\section{Quantum Mechanics Framework}

\subsection{Wave Function}
The fundamental wave function $\psi(x,t)$ describes the quantum state of a particle:
\begin{equation}
\psi(x,t) = A e^{i(kx - \omega t)}
\end{equation}

where:
\begin{itemize}
\item $A$ is the amplitude
\item $k$ is the wave number
\item $\omega$ is the angular frequency
\end{itemize}

\subsection{Schrödinger Equation}
The time-dependent Schrödinger equation governs the evolution of quantum systems:
\begin{equation}
i\hbar \frac{\partial \psi}{\partial t} = \hat{H} \psi
\end{equation}

where $\hat{H}$ is the Hamiltonian operator.

\section{Collaboration History}
\subsection{Initial Framework - Workflow Coordinator}
- Created basic quantum mechanics framework
- Established LaTeX structure for collaborative editing
- Set up version control integration

\section{Outstanding Questions}
1. How can we optimize the wave function representation for multi-particle systems?
2. What modifications to the Schrödinger equation might improve accuracy for relativistic particles?
3. Are there alternative mathematical frameworks that could provide better computational efficiency?

\section{Next Steps}
- Physics Expert AI 1: Review and suggest improvements to the quantum framework
- Physics Expert AI 2: Provide alternative approaches and comparative analysis
- Empirical validation of proposed modifications

\end{document}